\documentclass[10pt,journal,compsoc]{joser1}
\usepackage[pdftex]{graphicx}
\usepackage{epsfig}
\usepackage{amsfonts}
\usepackage{amsmath}
\usepackage{amsthm}
\usepackage{amssymb}
\usepackage[square,numbers]{natbib}
\usepackage{acronym}
\usepackage{dsfont}
\usepackage{pifont} % http://www.imsc.res.in/Computer/symbols-letter.pdf
\usepackage[pdftex]{thumbpdf}
\usepackage{multirow} % Prevent page breaks occurring within multi-line eqs.
\usepackage{rotating} % text rotation
\usepackage{booktabs} % rules in tables
% \usepackage{array} % alignment of cells in tables
\usepackage{calc}
\usepackage{verbatim}
\usepackage{units}
\usepackage{minted}
\usepackage{float} % separate pages for colour images
\usepackage{color}
% http://www.poeschko.com/2012/01/highlighting-changes-in-latex/
\usepackage{xcolor}
\usepackage[normalem]{ulem}
% \usepackage{trfsigns} % trfsigns.pdf
\usepackage{txfonts} % txfontsdoc.pdf
\usepackage{tikz}
\usepackage{pgfplots}
\usepackage{xmpincl}
\usepackage{eso-pic} % DRAFT
\usepackage{attachfile}

\definecolor{esored}{rgb}{1.00,0.95,0.95}
% http://de.narkive.com/2003/11/26/744404-wasserzeichen-mit-pdflatex.html
\AddToShipoutPicture{
  \put(0,0){
    \resizebox*{!}{\paperheight}{\rotatebox{57}{\hspace{1em}\textcolor{esored}{DRAFT}\hspace{1em}}}}}

% \usemintedstyle{trac}
\usemintedstyle{borland}
% \usemintedstyle{bw}

\definecolor{codegray}{rgb}{0.3,0.3,0.3}
\newcommand{\code}[1]{``\texttt{\textbf{\textcolor{codegray}{\small{#1}}}}''}

\newcommand{\sct}[1]{Section~\ref{cha:#1}}
\newcommand{\equ}[1]{Equation~\ref{equ:#1}}
\newcommand{\anx}[1]{Appendix~\ref{cha:#1}}
\newcommand{\fig}[1]{Figure~\ref{fig:#1}}
% \newcommand{\plt}[1]{Figure~\ref{fig:#1}}
\newcommand{\tbl}[1]{Table~\ref{tbl:#1}}
\newcommand{\lst}[1]{Listing~\ref{lst:#1}}

\acrodef{DSL}{domain specific language}
\acrodef{IRB}{Interactive Ruby Shell}
\acrodef{LOC}{lines of code}
\acrodef{MRI}{Matz's Ruby Interpreter}
\acrodef{REPL}{read-eval-print loop}

%%%%%%%%%%%%%%%%%%%%%%%%%%%%%%%%%%%%%%%%%%%%%%%%%%%%%%%%%%%%%%%%%%%%%%%
%%%%%%%%%%%%%%%%%%%%%%% will be inserted by the editor %%%%%%%%%%%%%%%%
%%%%%%%%%%%%%%%%%%%%%%%%%%%%%%%%%%%%%%%%%%%%%%%%%%%%%%%%%%%%%%%%%%%%%%%
\journalnumber{1}                       %will be inserted by the editor
\journalvolume{1}                       %will be inserted by the editor
\journalmonth{September}                %will be inserted by the editor
\journalyear{2009}                      %will be inserted by the editor
\articlefirstpage{123}                  %will be inserted by the editor
\articlelastpage{126}                   %will be inserted by the editor
\setcounter{page}{123}                  %will be inserted by the editor
%%%%%%%%%%%%%%%%%%%%%%%%%%%%%%%%%%%%%%%%%%%%%%%%%%%%%%%%%%%%%%%%%%%%%%%

\copyrightauthor{J. Wedekind, G. Chliveros, B. P. Amavasai}
\headoddname{J. WEDEKIND et al. / Computer Vision and Device Control using the Ruby Programming Language}

% correct bad hyphenation here
\hyphenation{
  in-vol-ving
  ma-na-ge-ment
}

\begin{document}
% paper title
\title{Computer Vision and Device Control\\\vskip 0.3\baselineskip using the Ruby Programming Language}

\author{
Jan Wedekind$^{1}$
\qquad
Georgios Chliveros$^{2}$
\qquad
Balasundram P. Amavasai$^{3}$

%%%%%%%%%%%%%%%%%%%%%%%%%%%%%%%%%%%%%%%%%%%%%%%%%%%%%%%%%%%%%%%%%%%%%%%
%%%%%%%%%%%%%%%%%%%%%%% will be inserted by the editor %%%%%%%%%%%%%%%%
%%%%%%%%%%%%%%%%%%%%%%%%%%%%%%%%%%%%%%%%%%%%%%%%%%%%%%%%%%%%%%%%%%%%%%%
\thanks{{\bf Regular paper} -- Manuscript received April 19, 2009;
revised July 11, 2009.}
%%%%%%%%%%%%%%%%%%%%%%%%%%%%%%%%%%%%%%%%%%%%%%%%%%%%%%%%%%%%%%%%%%%%%%%

\IEEEcompsocitemizethanks{\IEEEcompsocthanksitem This work was partially funded
  by the EPSRC Nanorobotics grant and a student bursary by the Materials and
  Engineering Research Institute at Sheffield Hallam University.\protect\\

  \IEEEcompsocthanksitem Authors retain copyright to their papers
  and grant JOSER unlimited rights to publish the paper electronically and in
  hard copy. Use of the article is permitted as long as the author(s) and the
  journal are properly acknowledged.}

} % end author

\address{
$^1$ 7 William Road, Sutton SM1 4QT, United Kingdom\\
$^2$ Institute of Computer Science, Foundation for Reasearch and
Technology - Hellas, Heraklion, Crete, Greece\\
$^3$ 96 Brooksby Road, Reading R31 6LY, United Kingdom}

% The paper headers
\markboth

% paper 6-10 pages, less than 5 MByte
\IEEEcompsoctitleabstractindextext{%
% No more than 500 words abstract
% define all nonstandard symbols and acronyms used
\begin{abstract}
  $\ldots$
\end{abstract}

\acresetall % repeat definitions even if introduced in abstract already

% 2-5 keywords from http://www.old.ieee-ras.org/uploads/tro/T-RO_Keywords.pdf
% or http://www.computer.org/portal/web/publications/acmtaxonomy
\begin{IEEEkeywords}
  Computer Vision,
  Image processing software,
  Input/output,
  Interactive systems,
  Reusable libraries,
  Very high-level languages.
\end{IEEEkeywords}}


% make the title area
\maketitle

\section{Introduction}
% The very first letter is a 2 line initial drop letter followed
% by the rest of the first word in caps.
\IEEEPARstart{M}{achine} vision is an important part of the interdisciplinary
field of robotics. Developers of programming systems in general and machine
vision software in particular face the challenge of bringing together
\begin{itemize}
  \item performance
  \item productivity
  \item generality
\end{itemize}
in a single architecture (see \fig{triangle}).
\begin{figure}[htbp]
   \begin{center}
     \resizebox{.5\columnwidth}{!}{\includegraphics{wedekind_20131110_f01}}\\
     \caption{The main requirements when designing a programming language or
     system \citep{wolczko2011}\label{fig:triangle}}
   \end{center}
\end{figure}

Scripting languages offer a great productivity advantage and the productivity
of programmer's measured in \ac{LOC} per hour seems to be independent of the
programming language\citep{prechelt2000empirical}. However performance is
critical when implementing a computer vision system. \emph{I.e.} when
prototyping a computer vision system using a scripting language such as Octave
(productivity \& generality), it is usually necessary to later port it to a
compiled language such as C/C++ (performance \& generality) to implement the
final system.

Today's desktop software is dominated by complex software systems which cannot
be understood by a single person any more. However most of this complexity can
be shown to be "accidental" rather than inherent\citep{ohshima2013kscript}.
Using \acp{DSL}, one can keep the size of software systems (measured in LOC)
\emph{orders of magnitudes} smaller and thus more
manageable\citep{kay2010steps}.

In this paper we are going to show how the Ruby programming language (which has
strong support for DSLs) can help to reduce accidental complexity when
implementing real-time computer vision software.

\section{Ruby Programming Language}
The Ruby programming language\footnote{\url{http://www.ruby-lang.org/}} is an
\emph{interpreted language}. It uses a \emph{mark-and-sweep} garbage collector
for memory management. Important features of the programming language are
\begin{itemize}
  \item \emph{late binding}
  \item \emph{closures}
  \item \emph{exception handling}
  \item \emph{continuations}
  \item \emph{open classes}
\end{itemize}

Ruby has support for \textbf{meta-programming}. In particular
\begin{itemize}
  \item \code{methods} returns an object's list of available methods
  \item \code{define\_method} defines a method at runtime
  \item \code{method\_missing} traps calls to nonexistent methods
  \item \code{send} dynamically calls a method
\end{itemize}

\ac{MRI} (the reference implementation of Ruby) comes with tool called the
\ac{IRB}. \ac{IRB} is a \ac{REPL} for the Ruby programming language. \lst{repl}
shows an example of an \ac{IRB} session. Comment lines (preceded with \code{\#}) show the output of the interpreter.
\begin{listing}[htbp]
  \begin{minted}[fontsize=\footnotesize]{ruby}
class Point
  attr_reader :x, :y
  def initialize x, y
    @x, @y = x, y
  end
  def inspect
    "Point(#{@x}, #{@y})"
  end
end
# nil
def Point x, y
  Point.new x, y
end
# nil
Point 1, 2
# Point(1, 2)
Point(1, 2).inspect
# "Point(1, 2)"
  \end{minted}
  \caption{The role of inspect methods when using the Interactive Ruby
  Shell\label{lst:repl}}
\end{listing}

The \ac{IRB} session shown in \lst{repl} illustrates an important feature of
Ruby: After evaluation the interpreter calls the \code{inspect} method of the
result in order to display it. \emph{I.e.} the user can control the way an
object is displayed by implementing the \code{inspect} method.

\lst{closure} demonstrates how closures facilitate the implementation of custom
control structures. The method \code{repeat} applies the closure \code{action}
$10$ times to the argument $1.0$. Note that the closure has access to
the variable \code{seed} while the method \code{repeat} does not.
\begin{listing}[htbp]
  \begin{minted}[fontsize=\footnotesize]{ruby}
def repeat n, x, &action
  if n.zero?
    x
  else
    repeat n - 1, action.call(x), &action
  end
end
# nil
seed = 2
# 2
repeat(10, 1.0) { |x| 0.5 * (x + seed / x) }
# 1.414213562373095
  \end{minted}
  \caption{Implementing custom control structure using a
  closure\label{lst:closure}}
\end{listing}

As mentioned earlier, Ruby implements \emph{open classes}. \emph{I.e.} after
having defined the class \code{Point} as shown in \lst{repl}, one can still add
methods (\emph{e.g.} scalar multiplication) to the class (see \lst{openclass}).
\begin{listing}[htbp]
  \begin{minted}[fontsize=\footnotesize]{ruby}
class Point
  def - c
    Point @x - c, @y - c
  end
end
# nil
Point(3, 4) - 2
# Point(1, 2)
  \end{minted}
  \caption{Ruby implements open classes\label{lst:openclass}}
\end{listing}

The Ruby standard library supports various numerical datatypes (\emph{e.g.}
complex numbers, rational numbers). It also implements \emph{coercions} to
facilitate binary operations involving custom datatypes. \lst{coerce} shows how
to implement \code{Point\#coerce} in order to implement binary operations where
the custom data type \code{Point} is the second argument.
\begin{listing}[htbp]
  \begin{minted}[fontsize=\footnotesize]{ruby}
class Point
  class Scalar
    def initialize c
      @c = c
    end
    def - point
      Point @c - point.x, @c - point.y
    end
  end
  def coerce c
    [Scalar.new(c), self]
  end
end
# nil
6 - Point(5, 4)
# Point(1, 2)
  \end{minted}
  \caption{Coercions for custom numeric datatypes\label{lst:coerce}}
\end{listing}

\section{Representing native types}
Ruby uses
\begin{itemize}
  \item \code{Fixnum} to represent small numbers
  \item \code{Bignum} to represent big numbers
\end{itemize}
as appropriate.  \emph{I.e.} when multiplying two small numbers, the result can
be a big number. However the standard library provides the methods
\code{Array\#pack} and \code{String\#unpack} in order to convert numerical
objects to a string with the native representation and back. \emph{E.g.}
\lst{pack} shows packing and unpacking of an 8-bit integer, a 32-bit signed
integer, and a single-precision floating-point number. The string \code{Cif} is
used to specify the native types to use when packing/unpacking.
\begin{listing}[htbp]
  \begin{minted}[fontsize=\footnotesize]{ruby}
s = [25, -123, Math::PI].pack('Cif')
# "\x19\x85\xFF\xFF\xFF\xDB\x0FI@"
s.unpack 'Cif'
# [25, -123, 3.1415927410125732]
  \end{minted}
  \caption{Packing/unpacking native representations of numerical
  objects\label{lst:pack}}
\end{listing}

While Ruby strings are sufficient to represent native data, they incur an
unnecessary overhead and in particular they don't support sharing of memory and
pointer operations (\emph{i.e.} sharing of data). Therefore a Ruby native
extension was implemented which defines the class \code{Malloc}. \code{Malloc}
objects allow one to see a chunk of memory as an array of cubbyholes, each
containing an 8-bit character (similar as \code{vector-ref} in
Scheme~\citep{abelson1996structure})\citep{thesis_wedekind}. Note that the
implementation records the size of each memory object and it also registers the
dependencies between the memory objects with the Ruby garbage collector. Thus
the implementation does not suffer from buffer overflows or dangling pointers.
Attempts to overrun the buffer's boundary are detected and an exception
(\emph{i.e.} \code{RuntimeError}) is raised as shown in \lst{malloc}.
\begin{listing}[htbp]
  \begin{minted}[fontsize=\footnotesize]{ruby}
m = Malloc.new 8
# Malloc(8)
m.write [2, 3, 5, 7, 11, 13, 17, 19].pack('C' * 8)
# "\x02\x03\x05\a\v\r\x11\x13"
(m + 5 * 1).read(1).unpack 'C'
# [13]
(m + 8 * 1).read(1).unpack 'C'
# RuntimeError: Only 0 bytes of memory left to read ...
  \end{minted}
  \caption{Introducing pointer arithmetic into Ruby\label{lst:malloc}}
\end{listing}

One can easily implement a class \code{INT} for representing native integers in
Ruby\citep{w2009}. By implementing the \code{inspect} method and by
implementing the operators \code{[]} and \code{[]=} one can achieve the
behaviour shown in \lst{wrap}.
\begin{listing}[htbp]
  \begin{minted}[fontsize=\footnotesize]{ruby}
i = INT.new
# INT(3)
i[] = 5
# 5
i
# INT(5)
i[]
# 5
  \end{minted}
  \caption{Behaviour of object representing a native integer\label{lst:wrap}}
\end{listing}
However as shown in \tbl{scalars} there are quite a few native scalar types and
composite types which potentially need to be supported. Therefore a more
generic approach is required.
\begin{table}[htbp]
  \begin{center}
    \caption{Native numerical types\label{tbl:scalars}}
    \begin{tabular}{llc}\toprule
    \textbf{type} &
    \textbf{description} &
    \parbox[t]{.2\columnwidth}{\textbf{mathematical definition}}\\\midrule
    BOOL     & boolean                 & $\mathbb{B}$\\
    UBYTE    & 8-bit unsigned integer  & $\{0,\ldots,2^8-1\}$\\
    BYTE     & 8-bit signed integer    & $\{-2^7,\ldots,2^7-1\}$\\
    USINT    & 16-bit unsigned integer & $\{0,\ldots,2^{16}-1\}$\\
    SINT     & 16-bit signed integer   & $\{-2^{15},\ldots,2^{15}-1\}$\\
    \multicolumn{1}{c}{$\vdots$} & \multicolumn{1}{c}{$\vdots$} & $\vdots$\\
    SFLOAT    & single-precision floating point & $\mathbb{R}$\\
    DFLOAT    & double-precision floating point & $\mathbb{R}$\\
    UBYTERGB & 8-bit unsigned RGB      & $\{0,\ldots,2^8-1\}^3$\\
    BYTERGB  & 8-bit signed RGB        & $\{-2^7,\ldots,2^7-1\}^3$\\
    USINTRGB & 16-bit unsigned RGB     & $\{0,\ldots,2^{16}-1\}^3$\\
    SINTRGB  & 16-bit signed RGB       & $\{-2^{15},\ldots,2^{15}-1\}^3$\\
    \multicolumn{1}{c}{$\vdots$} & \multicolumn{1}{c}{$\vdots$} & $\vdots$\\
    SFLOATRGB & single-precision RGB   & $\mathbb{R}^3$\\
    DFLOATRGB & double-precision RGB   & $\mathbb{R}^3$\\
    SCOMPLEX  & single-precision complex & $\mathbb{C}$\\
    DCOMPLEX  & double-precision complex & $\mathbb{C}$\\\bottomrule
    \end{tabular}
  \end{center}
\end{table}
% ------------------------------------------------------------------------------

% Type classes (template classes)
% Malloc objects

\begin{figure}[htbp]
  \begin{center}
    \resizebox{\columnwidth}{!}{\includegraphics{wedekind_20131110_f02}}\\
    \caption{Element types\label{fig:types}}
  \end{center}
\end{figure}


\begin{figure}[htbp]
   \begin{center}
     \resizebox{\columnwidth}{!}{\includegraphics{wedekind_20131110_f03}}\\
     \caption{Example of array type\label{fig:image}}
   \end{center}
\end{figure}


\begin{footnotesize}
\begin{itemize}
  \item down to the metal, instead of bindings
  \item unit testing,
  \item empirical evidence for productivity (LOC, structural complexity),
    reuse, maintainability
  \item theoretical contribution: generic operations, more fundamental
    understanding of computer vision
  \item (Const.new) MRI: single threaded (JRuby?, Rubinius?), meta-programming
    does not cover control statements (while, if, ...)
  \item Why not just ...?  Decision diagram: Lush, C++ class, C struct (OpenCV)
    \begin{itemize}
      \item simple method names (without types)
      \item only instantiate required/valid methods (unlike C++ templates)
      \item operations supported for arbitrary number of dimensions
      \item performance
      \item support for arbitrary type-combinations (in NArray/C++ not
        impossible but it is easier to work around those limitations)
    \end{itemize}
  \item value: Ruby object of matching type
  \item pointers (value: Malloc)
  \item variables, lambda, lookup, element-wise, ... (optimised building
    blocks) $\leftrightarrow$ Lambda calculus (minimal)
  \item Shared definitions for datatypes (images) are in Ruby and *not* in
    C/C++ $\rightarrow$ no static datatype imposed on the software, dynamic
    Ruby datastructures can be used in C/C++ using Ruby's native interface
  \item Integration (e.g. RMagick) using open classes,
    reuse/adaptation/reengineering of existing software artifacts.
  \item JIT compilation (using GCC): fetch, construct, store, coerce, coercion,
    descriptor, dup, compilable?, strip, get, assign, write, storage\_size,
    memory\_type, default, inspect
  \item fast development cycle
  \item maximally generic (binary operations (no subset of integer types),
    histograms, tensor operations, luts/warps, map (unary operation), inject,
    mask/unmask (larrabee), integral, convolution)
  \item Software architecture based on memory objects (Malloc) allows
    integration (no C++ "static" types).
  \item Requirements: generic operators on n-dimensional arrays, high
    performance, binary operations on any element-type (including array),
    generic operators instead of monolithic implementations
  \item Ruby allows for functional testing
  \item Using closures during camera initialisation and for displaying videos
  \item When appropriate, authors are encouraged to demonstrate the utility of
    their work on significant problems
  \item Integration of existing I/O libraries: ALSA, DC1394, FFMpeg, Kinect,
    OpenEXR, Qt4, RMagick, V4L2, Xorg
  \item Integration with other libraries: FFTW3, libswscale, linalg, narray,
    opencv
  \item Utility of the work: TODO
  \item Racket, Clojure, Factor, STEPS project
\end{itemize}
\end{footnotesize}

\section{Conclusion}
% Typical functions of the conclusion of a scientific paper include 1)
% summing up, 2) a statement of conclusions, 3) a statement of
% recommendations, and 4) a graceful termination. Any one of these, or
% any combination, may be appropriate for a particular paper. Some
% papers do not need a separate concluding section, particularly if
% the conclusions have already been stated in the introduction.

% \subsection{Initial Submission}
% Articles may be accompanied by online supplemental material.
% (Note: if an online appendix contains source code, we will require
% you to sign a release form prior to publication freeing us from
% liability.)
%
% \begin{enumerate}
%     \item a single source file author\_yyyymmdd.tex; do not include external
%     .tex files; use the submission date
%     \item the bibliography file author\_yyyymmdd.bib (only for BibTeX users)
%     \item the collection of figures as jpg or png files
%     author\_yyyymmdd\_f01.jpg, \_f02.jpg, etc.
%     \item the output file author\_yyyymmdd.pdf
% \end{enumerate}
%
% \section{Manuscript preparation}
% Manuscripts should include the following parts: title, authors$'$
% names, authors$'$ affiliations, abstract, keywords, text,
% acknowledgments (if necessary), collected references (e.g.
% ~\cite{IJSEK1996:Stewart, SEER2007:Vaughan, ROBIO2006:Friedmann,
% SIMPAR2008:Petters} and ~\cite{JARS2006:Colon, IJAR2001:Zielinski,
% IEEE-TSE1997:Stewart, ICIAS2008:Spexard,
% IROS2003:Montemerlo}), short
% biography and the photographs of every author (above 600 dpi),
% tables and figures (above 600 dpi). Supplemental material should
% be briefly described in section "Appendixes".
%
\section*{Acknowledgments}
The authors would like to thank...

% references section
\bibliographystyle{IEEEtran}
% argument is your BibTeX string definitions and bibliography database(s)
\bibliography{IEEEabrv,wedekind_20131110}


% biography section
\begin{IEEEbiography}[{wedekind_20131110_f06}]{Jan Wedekind}
received his B.\,Sc. and
M.\,Sc. degrees in mechanical engineering from the *** University,
in 1977 and 1984, respectively, and the Ph.\,D. degree in
computing from *** University, in 1992. In 1994, he was a
faculty member at *** University and in 1996 at ***
University. Currently, he is a professor in the Department of
Information System Engineering at *** University.
He has published about 100 refereed journal and conference papers.
His research interest covers robotics, software engineering, and distributed
systems.
Prof. Author received research award from Science Foundation, and
the Best Paper Award of the XX International Conference in 2000 and
2006, respectively. He is a member of ACM and IEEE.
\end{IEEEbiography}

\begin{IEEEbiography}[{wedekind_20131110_f05}]{Georgios Chliveros}
received his PhD in Pattern Recognition from Sheffield Hallam University (SHU)
in 2005. His undergraduate studies were on instrumentation systems and
his postgraduate studies on industrial statistics.
He was a researcher in robotics at the Materials and Engineering Research Institute, SHU
from 2004 to 2011. Currently he is a research engineer in machine vision and robotics at
the Foundation for Research and Technology Hellas (ICS-FORTH).
His research interests are on robotic systems and
computational engineering methods.
He is a member of the IET, UK and fellow of the RSS, UK.
\end{IEEEbiography}

\begin{IEEEbiography}[{wedekind_20131110_f04}]{Balasundram P. Amavasai}
received his B.\,Sc. and
M.\,Sc. degrees in mechanical engineering from the *** University,
in 1977 and 1984, respectively, and the Ph.\,D. degree in
computing from *** University, in 1992. In 1994, he was a
faculty member at *** University and in 1996 at ***
University. Currently, he is a professor in the Department of
Information System Engineering at *** University.
He has published about 100 refereed journal and conference papers.
His research interest covers robotics, software engineering, and distributed
systems.
Prof. Author received research award from Science Foundation, and
the Best Paper Award of the XX International Conference in 2000 and
2006, respectively. He is a member of ACM and IEEE.
\end{IEEEbiography}

% insert where needed to balance the two columns on the last page with
% biographies

% You can push biographies down or up by placing
% a \vfill before or after them. The appropriate
% use of \vfill depends on what kind of text is
% on the last page and whether or not the columns
% are being equalized.

\vfill

% Can be used to pull up biographies so that the bottom of the last one
% is flush with the other column.
%\enlargethispage{-5in}

% that's all folks
\end{document}
